\documentclass{pgnotes}

\title{Database Management Systems}

\begin{document}

\maketitle

\section{Database management systems (DBMS)}
\label{sec:dbms}

For overall history see \citep{grad:2009:history}. 

\subsection{Client-server}

Most database management systems run in a client-server model, \autoref{fig:dbms-client-server}.

\autoimage{dbms_client_server}{Client-server DBMS}{dbms-client-server}

The server process manages the data store and processes requests from clients.
The server can be running on any of the following \textit{hosts}:
\begin{itemize}
\item Standard laptop / desktop computer
\item Dedicated server computer (in a data centre environment)
\item Cloud-based virtual host, called a compute instance. (e.g. Amazon EC2) 
\item A managed database service provided by a cloud service provider (e.g. Amazon RDS, Azure, Google Cloud, IBM Cloud)
\end{itemize}

The client program accesses the server using a server-specific protocol.
Clients normally access through IP networks using TCP on a specified port number.
Examples of clients:
\begin{itemize}
\item Most databases have a simple command-line client that can send requests to the database and display results
\item Apps can be written to access database servers using a client library.
  \begin{itemize}
  \item Generally the text-mode client uses this library internally too! 
  \end{itemize}
\end{itemize}

Two things to note about the client:
\begin{itemize}
\item The client may in some cases be running on the same host as the server.
\item Software that is the client of a DBMS may itself be a server.
  \begin{itemize}
  \item Example: a web application is written in Java using the Spring framework and provides a web server using an embedded Tomcat server. The web application is itself a client of the DBMS it accesses.
  \end{itemize}
\end{itemize}

This also implies that there is a degree of concurrency, where multiple clients access the same database at the same time.

\autoimage{dbms_concurrent_access}{Concurrent access to a DBMS hosting a college timetable}{dbms-concurrent-access}

\section{Key terms}




\section{Structured query language (SQL)}
\label{sec:sql}

For a general overview see \citep{chamberlain:2012:early}.


\section{PostgreSQL}
\label{sec:postgresql}

We will focus on PostgreSQL as our primary database.
Later on we will introduce other technologies. 
Reasons:
\begin{itemize}
\item Support exists for geospatial data, JSON, XML, full-text search etc.
\item It is free software and can be installed on any operating system.
\end{itemize}
You should bookmark the \href{https://www.postgresql.org/docs/13/}{PostgreSQL} documentation.

As we continue we will refer to PostgreSQL as Postgres for brevity.


\subsection{Connecting to a remote host}

\autoimage{ssh_psql_usage}{Using psql on a remote host over SSH}{ssh-psql-usage}



\bibliographystyle{plainnat}
\bibliography{bibliography}


\end{document}


